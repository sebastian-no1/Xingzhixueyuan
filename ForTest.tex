%%%%%%%%%%%%%%%%%%%%%%%%%%%%%%%%%%%%%%%%%
% Sullivan Business Report
% LaTeX Template
% Version 1.0 (May 5, 2022)
%
% This template originates from:
% https://www.LaTeXTemplates.com
%
% Author:
% Vel (vel@latextemplates.com)
%
% License:
% CC BY-NC-SA 4.0 (https://creativecommons.org/licenses/by-nc-sa/4.0/)
%
%%%%%%%%%%%%%%%%%%%%%%%%%%%%%%%%%%%%%%%%%

%----------------------------------------------------------------------------------------
%	CLASS, PACKAGES AND OTHER DOCUMENT CONFIGURATIONS
%----------------------------------------------------------------------------------------

\documentclass[
	a4paper, % Paper size, use either a4paper or letterpaper
	12pt, % Default font size, the template is designed to look good at 12pt so it's best not to change this
	%unnumberedsections, % Uncomment for no section numbering
]{CSSullivanBusinessReport}
\addbibresource{sample.bib} % BibLaTeX bibliography file
%----------------------------------------------------------------------------------------
%	REPORT INFORMATION
%----------------------------------------------------------------------------------------
\reporttitle{Student Entry Exam} % The report title to appear on the title page and page headers, do not create manual new lines here as this will carry over to page headers

\reportsubtitle{This document is licensed under \\行知学園 } % Report subtitle, include new lines if needed

\reportauthors{Created by:行知学園講師--    HE YAN\\\smallskip Vel (sebastian.y.aa@m.titech.ac.jp)} % Report authors/group/department, include new lines if needed

\reportdate{\today} % Report date, include new lines for additional information if needed

\rightheadercontent{\includegraphics[width=3cm]{logo.eps}} % The content in the right header, you may want to add your own company logo or use your company/department name or leave this command empty for no right header content
\usepackage{wallpaper}
\usepackage{transparent}
\usepackage{eso-pic,lipsum}        % 导入包
\newcommand\BackgroundPic
{%
 	\put(150,-320)
	   {              % 调整图片的未知
		\parbox[b][\paperheight]{\paperwidth}
		{%
			\vfill
			\centering
			    \includegraphics[ width=4in,angle=0, % 设置图片宽度,旋转15度
			 keepaspectratio]{background.eps}%
			\vfill  
		}
	}
}

%\usepackage{background}
%\backgroundsetup{scale=2, angle=0, opacity = 1,contents = {\includegraphics[width=\paperwidth, height=\paperwidth, keepaspectratio]{Background.eps}}}
%----------------------------------------------------------------------------------------
\usepackage{hyperref}
\hypersetup{
					colorlinks=true,
					linkcolor=blue,
					urlcolor=cyan,
					citecolor=green,
					pdfpagemode=FullScreen
				}
\urlstyle{same}

\begin{document}


\AddToShipoutPicture*{\BackgroundPic} 


%----------------------------------------------------------------------------------------
%	TITLE PAGE
%----------------------------------------------------------------------------------------
%\CenterWallPaper{1}{Background.eps}
\thispagestyle{empty} % Suppress headers and footers on this page

\begin{fullwidth} % Use the whole page width
	\vspace*{-0.075\textheight} % Pull logo into the top margin
	
	\hfill\includegraphics[width=5cm]{logo.eps} % Company logo

	\vspace{0.15\textheight} % Vertical whitespace

	\parbox{0.9\fulltextwidth}{\fontsize{50pt}{52pt}\selectfont\raggedright\textbf{\reporttitle}\par} % Report title, intentionally at less than full width for nice wrapping. Adjust the width of the \parbox and the font size as needed for your title to look good.
	
	\vspace{0.03\textheight} % Vertical whitespace
	
	{\LARGE\textit{\textbf{\reportsubtitle}}\par} % Subtitle
	
	\vfill % Vertical whitespace
	
	{\Large\reportauthors\par} % Report authors, group or department
	
	\vfill\vfill\vfill % Vertical whitespace
	
	{\large\reportdate\par} % Report date
\end{fullwidth}
\newpage

%----------------------------------------------------------------------------------------
%	DISCLAIMER/COPYRIGHT PAGE
%----------------------------------------------------------------------------------------

\thispagestyle{empty} % Suppress headers and footers on this page

\begin{twothirdswidth} % Content in this environment to be at two-thirds of the whole page width
	\footnotesize % Reduce font size
	
	\subsection*{Disclaimer}

This document is endorsed and provided by XingZhiXueYuan, for educational purposes only. All information on this document is provided in good faith, however, we make no representation or warranty of any kind, express or implied, regarding the accuracy, adequacy, validity, reliablity, availablity or completeness of any information on this material.

	\subsection*{Copyright}
		
	Typeset in \LaTeX\ 
	
	\textcopyright~[2023] [行知学園] 
	
	All right reserved.

	
	\subsection*{Contact}
	
	Address Line 1\\
	Address Line 2\\
	Address Line 3
	
	Business Number 123456
	
	Contact: name@company.com
	
	\vfill % Push the following down to the bottom of the page
	
	\subsubsection*{Changelog}
	
	\scriptsize % Reduce font size further
	
	\begin{tabular}{@{} L{0.05\linewidth} L{0.15\linewidth} L{0.6\linewidth} @{}} % Column widths specified here, change as needed for your content
		\toprule
		%v1.0 & 20XX-02-05 & Lorem ipsum dolor sit amet, consectetur adipiscing elit. Praesent porttitor arcu luctus, imperdiet urna iaculis, mattis eros.\\
		%v1.1 & 20XX-02-27 & Pellentesque iaculis odio vel nisl ullamcorper, nec faucibus ipsum molestie.\\
		v1.0  & 2023-12-13 & First time creation.\\
		\bottomrule
	\end{tabular}
\end{twothirdswidth}

\newpage

%----------------------------------------------------------------------------------------
%	TABLE OF CONTENTS
%----------------------------------------------------------------------------------------

\begin{twothirdswidth} % Content in this environment to be at two-thirds of the whole page width
	\tableofcontents % Output the table of contents, automatically generated from the section commands used in the document
\end{twothirdswidth}

\newpage

%----------------------------------------------------------------------------------------
%	SECTIONS
%----------------------------------------------------------------------------------------
\begin{fullwidth} 
\section{数学} % Top level section
\subsection{フーリエ解析}



\subsection{複素数概論}



\subsection{ベクトル解析}

\subsection{偏微分方程式}



\subsection{常微分方程式}



\section{材料力学}
\textbf{問題1}

機械類は、使用時にかかる引張荷重に対して部材の強度が十分となるように設計される。また、荷重
が解消された後も、ひずみが残らないようにしている。安全が保証される引張荷重の最大値(ここまでの
荷重ならば、かけても十分に安全)を___という。%許容応力 σa [Pa]

\textbf{問題2}

図 2.3.1 のように,長さが  の棒の三か所に荷重が作用している.軸力分布図を描きなさい

\begin{center}
	\centering 
  \includegraphics[width=0.5\linewidth]{2.eps} 
  \label{fig:2}
\end{center}

\textbf{問題 3}

図 2.3.7 のように,長さが 1.0m の棒の四か所に荷重が作用している.軸力分布図を描きなさい.
\begin{center}
	\centering 
  \includegraphics[width=0.5\linewidth]{3.eps} 
  \label{fig:2}
\end{center}

\textbf{問題 4}

\begin{center}
	\centering 
  \includegraphics[width=0.8\linewidth]{4.eps} 
  \label{fig:2}
\end{center}

\textbf{問題 5}

\begin{center}
	\centering 
  \includegraphics[width=0.8\linewidth]{5.eps} 
  \label{fig:2}
\end{center}


\textbf{問題 6}

\begin{center}
	\centering 
  \includegraphics[width=0.8\linewidth]{6.eps} 
  \label{fig:2}
\end{center}





\section{機械力学}



\newpage
\section{熱力学}

以下の循環の熱効率を求め、それぞれの過程を明記し、計算過程も残しなさい


1.1 カルノーサイクル、P-V図も書きなさい
\\[200pt]
1.2 オットーサイクル、P-V図とT-S図も書きなさい、圧縮比は$ \varepsilon $とする
\\[200pt]
1.3 ディーゼルサイクル、P-V図とT-S図も書きなさい、圧縮比は$ \varepsilon $とする、締切比は$ \rho $とする
\\[200pt]
\section{流体力学}
\textbf{問題 1}

以下の定義をそれぞれ説明しなさい。(図を使っても良い)

流跡線:

流脈線:

流線:

レイノルズ数:


\section{制御工学}





%----------------------------------------------------------------------------------------
\end{fullwidth}
\end{document}
